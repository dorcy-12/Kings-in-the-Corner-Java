\chapter{Nicht-funktionale Anforderungen}

\newcounter{nf}\setcounter{nf}{10}

\section{Softwarearchitektur}

\begin{description}[leftmargin=5em, style=sameline]	
	\begin{lhp}{nf}{NF}{nfunk:sarch1}
		\item [Name:] Client-Server Anwendung
		\item [Beschreibung:] Das verteilte Spiele-System ermöglicht das gemeinsame Spielen von verschiedenen Rechnern aus.
		\item [Motivation:] Aufgabestellung v. SEP.
		\item [Erfüllungskriterium:] Das fertige System besteht aus Client- und Server-Teilen.
	\end{lhp}
	
	\begin{lhp}{nf}{NF}{nfunk:sarch1}
		\item [Name:] Plattformunabhängigkeit
		\item [Beschreibung:] Es soll sich um eine plattformunabhängige Anwendung handeln. Zumindest Windows-, Linuxsysteme und MAC-OS sind zu unterstützen.
		\item [Motivation:] Aufgabenstellung v. SEP.
		\item [Erfüllungskriterium:] Es soll ein flüssiges Spiel möglich sein auf allen Systemen, ohne zu lange Wartezeiten. Entwicklung einer kompilierten Version für Windows-, Linuxsysteme und MAC-OS.
	\end{lhp}
\end{description}



\section{Benutzerfreundlichkeit}


\begin{description}[leftmargin=5em, style=sameline]	
	\begin{lhp}{nf}{NF}{nfunk:alter}
		\item [Name:] Benutzeralter
		\item [Beschreibung:] Das System ist für Benutzer geeignet, die älter als 5 Jahre sind.
		\item [Motivation:] Jüngere Benutzer sind unfähig das Spiel zu spielen.
		\item [Erfüllungskriterium:] In den AGBs steht ein entsprechender Hinweis.
	\end{lhp}
\end{description}

\begin{description}[leftmargin=5em, style=sameline]	
	\begin{lhp}{nf}{NF}{nfunk:keinetechniker}
		\item [Name:] Technische Fähigkeiten
		\item [Beschreibung:] Besondere technische Fähigkeiten sind von den Benutzern nicht zu erwarten.
		\item [Motivation:] Auch die Menschen, die kaum etwas von Bedienung bzw. Programmierung von Rechnern verstehen, sollen fähig sein, das System zu verwenden.
		\item [Erfüllungskriterium:] Das Spiel kann fachfremden Personen vorgeführt werden und diese können das Spiel ohne Probleme installieren und spielen. Leicht zu bedienende Oberfläche und verwendete Begriffe müssen allgemein verständlich sein.
	\end{lhp}
\end{description}

\section{Leistungsanforderungen}

\begin{description}[leftmargin=5em, style=sameline]	
	\begin{lhp}{nf}{NF}{nfunk:antwortzeit}
		\item [Name:] Antwortzeit
		\item [Beschreibung:] Maximale Antwortzeit für alle Systemprozesse.
		\item [Motivation:] Das System muss immer brauchbar sein.
		\item [Erfüllungskriterium:] Das System antwortet auf Benutzerhandlungen nie später als in 10 Sekunden.
	\end{lhp}
\end{description}

\section{Anforderungen an Einsatzkontext}

\subsection{Anforderungen an physische Umgebung}

\begin{description}[leftmargin=5em, style=sameline]	
	\begin{lhp}{nf}{NF}{nfunk:beispiel1}
		\item [Name:] Lauffähigkeit an SCI-Rechnern
		\item [Beschreibung:] Das Produkt muss auf einem eigenem Gerät lauffähig sein, welches zur Präsentation am Ende des SEP genutzt werden muss. Falls keine eigenen Rechner vorhanden sind, stehen auch die SCI-Terminals zur Verfügung.
		\item [Motivation:] Optimierung von Betreuung und Abnahme des SEP
		\item [Erfüllungskriterium:] Es ist ein Rechner verfügbar, der das Programm ausführen kann.
	\end{lhp}
\end{description}


%\subsection{Anforderungen an benachbarte Systeme}
%(sehe Systemkontext)
%
%\begin{description}[leftmargin=5em, style=sameline]	
%	\begin{lhp}{nf}{NF}{nfunk:beispiel2}
%		\item [Name:] Beispiel
%		\item [Beschreibung:] 
%		\item [Motivation:] 
%		\item [Erfüllungskriterium:] 
%	\end{lhp}
%\end{description}

\subsection{Absatz- sowie Installationsbezogene Anforderungen}

\begin{description}[leftmargin=5em, style=sameline]	
	\begin{lhp}{nf}{NF}{nfunk:beispiel3}
		\item [Name:] Installationsanleitung	
		\item [Beschreibung:] Falls die Installation nicht lediglich das Öffnen einer Datei voraussetzt, muss der genaue Installations- und Startvorgang schriftlich für Benutzer zur Verfügung gestellt werden.
		\item [Motivation:] Spezifikation
		\item [Erfüllungskriterium:] In den Programmdateien ist eine Installationsanleitung als Dokument enthalten.
	\end{lhp}
\end{description}

\subsection{Anforderungen an Versionierung}

\begin{description}[leftmargin=5em, style=sameline]	
	\begin{lhp}{nf}{NF}{nfunk:beispiel4}
		\item [Name:] Keine weitere Versionen
		\item [Beschreibung:] Nach Version 1.0 ist keine weitere Entwicklung vorgesehen.
		\item [Motivation:] Das ist nur das SEP, kein Geschäftsprojekt, siehe \ref{fa:fortentwicklung}
		\item [Erfüllungskriterium:] Kein Entwickler entwickelt im Namen aller das Projekt weiter und veröffentlicht neue Versionen nach der Abnahme.
	\end{lhp}
\end{description}

\section{Anforderungen an Wartung und Unterstützung}

\subsection{Wartungsanforderungen}

\begin{description}[leftmargin=5em, style=sameline]	
	\begin{lhp}{nf}{NF}{nfunk:beispiel4}
		\item [Name:] Hiwibemängelung
		\item [Beschreibung:] Falls der Hiwi während des Projektes Mängel feststellt, werden diese von uns Zeitnah behoben. 
		\item [Motivation:] Die Vorgaben des Projektes zufriedenstellend zu erfüllen.
		\item [Erfüllungskriterium:] Alle Mängel, die die Hiwis feststellen, werden Zeitnah behoben
	\end{lhp}
\end{description}

\begin{description}[leftmargin=5em, style=sameline]	
	\begin{lhp}{nf}{NF}{nfunk:doku}
		\item [Name:] Dokumentation
		\item [Beschreibung:] Der Quellcode muss ausführlich dokumentiert werden.
		\item [Motivation:] Das Projekt bleibt dadurch übersichtlich.
		\item [Erfüllungskriterium:] JavaDoc 
	\end{lhp}
\end{description}

\begin{description}[leftmargin=5em, style=sameline]	
	\begin{lhp}{nf}{NF}{nfunk:doku}
		\item [Name:] Testen
		\item [Beschreibung:] Der Quellcode außer GUI muss gut getestet werden.
		\item [Motivation:] Möglichst keine Fehler im Quellcode zu haben, die das Spiel zum abstürzen bringen.
		\item [Erfüllungskriterium:] Von Unit-Tests muss mindestens 70\% des Quellcodes bedeckt werden. GUI-Klassen sind aus der Anforderung ausgenommen.
	\end{lhp}
\end{description}

\subsection{Anforderungen an technische und fachliche Unterstützung}

\begin{description}[leftmargin=5em, style=sameline]	
	\begin{lhp}{nf}{NF}{nfunk:kundensupport}
		\item [Name:] Kundensupport
		\item [Beschreibung:] Es ist keine technische und fachliche Unterstützung des Systems geplant.
		\item [Motivation:] Siehe \ref{fa:fortentwicklung}.
		\item [Erfüllungskriterium:] Nicht anwendbar.
	\end{lhp}
\end{description}

\subsection{Anforderungen an technische Kompatibilität}

\begin{description}[leftmargin=5em, style=sameline]	
	\begin{lhp}{nf}{NF}{nfunk:plattformunabhängigkeit}
		\item [Name:] Plattformunabhängigkeit
		\item [Beschreibung:] Es soll sich um eine plattformunabhängige Anwendung handeln. Zumindest Windows-, Linuxsysteme und MAC-OS sind zu unterstützen.
		\item [Motivation:] Aufgabenstellung v. SEP.
		\item [Erfüllungskriterium:] Es soll ein flüssiges Spiel möglich sein auf allen Systemen, ohne zu lange Wartezeiten. Entwicklung einer kompilierten Version für Windows-, Linuxsysteme und MAC-OS.
	\end{lhp}
\end{description}

\section{Sicherheitsanforderungen}

\subsection{Zugang}

\begin{description}[leftmargin=5em, style=sameline]	
	\begin{lhp}{nf}{NF}{nfunk:passwortgesicherter_login}
		\item [Name:] Passwortgesicherter Login
		\item [Beschreibung:] Zu jedem Konto gehört ein Passwort.
		\item [Motivation:] Accountsicherheit
		\item [Erfüllungskriterium:] Das Passwort muss bestimmte Kriterien erfüllen wie eine bestimmte Mindestlänge und Verwendung bestimmter Satzzeichen.
	\end{lhp}
\end{description}

\subsection{Integrität}

\begin{description}[leftmargin=5em, style=sameline]	
	\begin{lhp}{nf}{NF}{nfunk:transparenz}
		\item [Name:] Transparenz
		\item [Beschreibung:] In den AGBs Aufführung der Arten, wie die Daten der Nutzer verarbeitet werden.
		\item [Motivation:] Informieren der Nutzer über die Verwendung ihrer Daten
		\item [Erfüllungskriterium:] Erstellen eines AGB Paragraphen der auf diese Fragestellung eingeht.
	\end{lhp}
\end{description}

\subsection{Datenschutz/Privatsphäre}

\begin{description}[leftmargin=5em, style=sameline]	
	\begin{lhp}{nf}{NF}{nfunk:geheimhaltung_der_nutzerdaten}
		\item [Name:] Geheimhaltung der Nutzerdaten
		\item [Beschreibung:] Nutzerdaten werden nicht veröffentlicht und mit besten Gewissen behandelt.
		\item [Motivation:] Datenschutz und Schutz der Privatsphäre der Nutzer.
		\item [Erfüllungskriterium:] Nutzerdaten werden gespeichert und nicht weitergegeben.
	\end{lhp}
\end{description}


\subsection{Virenschutz}

\begin{description}[leftmargin=5em, style=sameline]	
	\begin{lhp}{nf}{NF}{nfunk:virenschutz}
		\item [Name:] virenschutz
		\item [Beschreibung:] Daten der Spieler sollen sicher sein vor unbefugten Zugriff durch Dritte.
		\item [Motivation:] Es gibt kein Virenschutz, da das Spiel nicht veröffentlicht wird.
		\item [Erfüllungskriterium:] Kein Virenschutz implementiert.
	\end{lhp}
\end{description}

\section{Prüfungsbezogene Anforderungen}

Anforderungen, die sich auf die Prüfung/Audit vom System von SEP-Tutoren oder von weiteren Instanzen beziehen.


\begin{description}[leftmargin=5em, style=sameline]	
	\begin{lhp}{nf}{NF}{nfunk:beispiel10}
		\item [Name:] Formate der Systemdokumentation
		\item [Beschreibung:] Systemdokumantation muss in 2 Formen geführt werden (wenn anwendbar): Die Ausgangsdateien (\LaTeX, Dateien der Diagrammerstellungssoftware, Dateien der Grafiksoftware usw.) und PDFs.
		\item [Motivation:] Optimierung der SEP-Betreuung.
		\item [Erfüllungskriterium:] Siehe Beschreibung.
	\end{lhp}
\end{description}

\section{Kulturelle und politische Anforderungen}


\begin{description}[leftmargin=5em, style=sameline]	
	\begin{lhp}{nf}{NF}{nfunk:beispiel11}
		\item [Name:] Systemsprache
		\item [Beschreibung:] Die Interfacesprache ist Deutsch.
		\item [Motivation:] Synchronisation des Verständnisses von Teammitgliedern mit unterschiedlichen kulturellen Hintergründen.
		\item [Erfüllungskriterium:] Alle sichtbaren Anwendungen sind in der Deutschen Sprache verfasst genauso wird die Dokumentation des Projektes ebenfalls in der deutschen Sprache sein.
	\end{lhp}
\end{description}

\section{Rechtliche und standardsbezogene Anforderungen}


\begin{description}[leftmargin=5em, style=sameline]	
	\begin{lhp}{nf}{NF}{nfunk:beispiel12}
		\item [Name:] Nicht rechtliche Anforderungen
		\item [Beschreibung:] Keine relevanten rechtlichen Anforderungen bekannt.
		\item [Motivation:] Siehe \ref{fa:fortentwicklung}.
		\item [Erfüllungskriterium:] Nicht anwendbar.
	\end{lhp}
\end{description}
